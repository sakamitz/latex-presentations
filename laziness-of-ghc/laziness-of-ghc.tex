\documentclass{beamer}

\usepackage{fontspec}
\usepackage[utf8]{inputenc}
\usepackage[cache=false]{minted}
\usetheme{Madrid}

\setmonofont{Fira Code}
\newmintedfile[haskellcode]{haskell}{style=trac}
\AtBeginSection[]
{
  \begin{frame}
    \frametitle{Contents}
    \tableofcontents[currentsection]
  \end{frame}
}
%--------------------------------------------

\title[Laziness in GHC Haskell]
{Laziness in GHC Haskell}

\subtitle{The features and principles}

\author[chip]
{Presented by chip}

\institute[ZJU]
{
  ZJU Lambda\\
  From here to World
}

\date[ZJU-Lambda 2019]
{ZJU-Lambda Conference, May 2019}

\logo{\includegraphics[height=1.2cm]{./pic/haskell-logo.png}}

\begin{document}
\frame{\titlepage}


\section{Appetizer}
%--------------------------------------------

\begin{frame}
\frametitle{Course 1: Outside in}

\haskellcode{src/outside-in.hs}
\par\noindent\rule{0.75\textwidth}{1.0pt}
\newline\newline
If we apply \mintinline{haskell}{possiblyBottom} to \mintinline{haskell}{True}, we will get a \mintinline{haskell}{0}.

\end{frame}

%--------------------------------------------

\begin{frame}
\frametitle{Course 1: Outside in}

A slightly arcane form:\newline
\haskellcode{src/arcane-form.hs}

\end{frame}

%--------------------------------------------

\begin{frame}
\frametitle{Course 1: Outside in}

Nesting lambdas and reducing from the outside in:\newline
(They are not in fact decomposed this way by the compiler)\newline
\haskellcode{src/lambda-nesting.hs}

\end{frame}

%--------------------------------------------

\begin{frame}
\frametitle{Course 2: Evaluate to WHNF}

\haskellcode{src/evaluate-to-WHNF.hs}
\par\noindent\rule{0.7\textwidth}{1.0pt}
\newline\newline
It prints \mintinline{haskell}{2} !
\newline
What happened here?

\end{frame}

%--------------------------------------------

\begin{frame}
\frametitle{Example 2: Evaluate to WHNF}
The actual evaluation process:
\newline
\haskellcode{src/length'-procedure.hs}

\begin{block}{Concept}
In WHNF, we only evaluate the outermost constructor
\end{block}
\end{frame}


\section{Thunk? What's it?}
%--------------------------------------------

\begin{frame}
\frametitle{The Haskell Heap}
\begin{center}
    The Haskell heap is a rather strange place.
\end{center}
\begin{figure}[hbt!]
    \centering
    \includegraphics[height=0.5\textheight]{./pic/haskell-heap.png}
\end{figure}
\end{frame}

%--------------------------------------------

\begin{frame}
\frametitle{Box}
Every item is wrapped up nicely in a box:\newline
The Haskell heap is a heap of \textit{presents} (thunks).
\begin{figure}[hbt!]
    \centering
    \includegraphics[height=0.4\textheight]{./pic/thunk.png}
\end{figure}
\end{frame}

%--------------------------------------------

\begin{frame}
\frametitle{Present}
When you actually want what’s inside the present, you \textit{open it up} (evaluate it).\newline
\begin{figure}[hbt!]
    \centering
    \includegraphics[height=0.4\textheight]{./pic/thunk-nullary.png}
\end{figure}
\end{frame}

%--------------------------------------------

\begin{frame}
\frametitle{Gift card}
Sometimes you open a present, you get a \textit{gift card} (data constructor).\newline
Gift cards have two traits.\newline
\begin{itemize}
    \item A name. (the \textbf{Just} gift card or \textbf{Right} gift card)\newline
    \item And they tell you where the rest of your presents are.\newline
\end{itemize}
There might be more than one (the tuple gift card), if you’re a lucky duck!
\begin{figure}[hbt!]
    \centering
    \includegraphics[height=0.4\textheight]{./pic/thunk-constructor.png}
\end{figure}
\end{frame}

%--------------------------------------------

\begin{frame}
\frametitle{Tricksters}
Presents on the Haskell heap are rather mischievous.\newline

\begin{columns}
\column{0.5\textwidth}
\begin{figure}[hbt!]
    \includegraphics[height=0.3\textheight]{./pic/thunk-bomb.png}
\end{figure}
\begin{center}
Explode when you open it
\end{center}

\column{0.5\textwidth}
\begin{figure}[hbt!]
    \includegraphics[height=0.3\textheight]{./pic/thunk-ghost.png}
\end{figure}
\begin{center}
Haunted by ghosts that open other presents when disturbed
\end{center}
\end{columns}
\end{frame}

%--------------------------------------------

\begin{frame}
\frametitle{What is a \textit{thunk}?}
\mintinline{haskell}{<thunk: expression-to-be-evaluated>}\bigskip
\begin{itemize}
    \item A box containing unevaluated expressions.
    \item Being evaluated when \textit{needed}.
    \item Basically \textbf{anything} creates a thunk in (GHC) Haskell, by default
\end{itemize}
\end{frame}

%--------------------------------------------

\begin{frame}
\frametitle{Example: Evaluate a thunk}
How will this expression be evaluated?\newline\bigskip
\mintinline{haskell}{map negate [1,2,3]}\newline\bigskip\pause
\mintinline{haskell}{<@: map negate <@: (1:2:3:[])>>}\newline\bigskip\pause
\mintinline{haskell}{<@: negate <@: 1> : <@: map negate <@: [2,3]>>}\newline\bigskip\pause
\mintinline{haskell}{-<@: 1> : <@: map negate <@: [2,3]>>}\newline\bigskip\pause
\mintinline{haskell}{-1 : <@: map negate <@: [2,3]>>}
\end{frame}

%--------------------------------------------

\begin{frame}
\frametitle{Thunk brings us...}
\begin{itemize}
    \item On-demand data types.
    \item Call-by-need strategy.
    \item Memory reuse on CAF (Constant Applicative Forms).
    \item ...
\end{itemize}\bigskip
\haskellcode{src/fibs.hs}
\end{frame}


\section{Why we need strictness?}
%--------------------------------------------

\begin{frame}
\frametitle{Thunks are good, but...}
\haskellcode{src/lazy-foldl.hs}\bigskip
What about \mintinline{haskell}{foldl (+) 0 [1..1000000000]} ?\newline
\begin{figure}[hbt!]
    \centering
    \includegraphics[height=0.4\textheight]{./pic/evil-of-thunk.png}
\end{figure}
\end{frame}

%--------------------------------------------

\begin{frame}
\frametitle{Memory leak}
After executing \mintinline{haskell}{foldl (+) 0 [1..1000000000]}\newline
\begin{figure}[hbt!]
    \centering
    \includegraphics[height=0.1\textheight]{./pic/memory-usage-ghc.png}
    \newline\newline
    A veritable ghost jamboree in our memory!
    \includegraphics[height=0.4\textheight]{./pic/ghost-party.png}
\end{figure}
\end{frame}

\end{document}